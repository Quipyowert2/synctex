%!TEX TS-program = xetex
%!TEX encoding = UTF-8 Unicode
% typeset from command line with
%   xetex -papersize=512,384 xetex-wuhan
% to get correct-sized media

\TeXXeTstate=1
\hoffset=-1in
\voffset=-1in
\newdimen\width \width=512bp
\newdimen\height \height=384bp

\def\plainoutput{\shipout\vbox to \height{\hrule\vss
 \hbox to \width{\vrule \hfil
  \vbox to \height {\makeheadline \pagebody \makefootline \vfil}\hfil\vrule}\vss\hrule}
 \advancepageno \ifnum 
 \outputpenalty >-20000 \else \dosupereject \fi}

\def\={\output={\partialslideout}\vfill\eject \output={\plainoutput}}
\def\partialslideout{\setbox\savepage=\vbox{\unvcopy255\unskip}\plainoutput
 \global\output={\plainoutput}\unvbox\savepage}
\newbox\savepage

\let\=\relax % comment out this to get progressively-revealed slides

\raggedbottom
\frenchspacing

\vsize=0.91\height
\dimen0=\smallskipamount
\parskip=\dimen0
\rightskip=0pt plus 2in
\hyphenpenalty=1000
\catcode`\_=12

\def\fontfam{Brioso Pro}

\font\A="\fontfam/S=18:color=000066;mapping=tex-text" at 36pt
\font\B="\fontfam/S=14:color=660000;mapping=tex-text" at 24pt
\font\C="\fontfam/S=11:mapping=tex-text" at 20pt
\font\Cw="\fontfam/S=11:color=FFFFFF;mapping=tex-text" at 20pt
\font\D="\fontfam/S=9" at 16pt
\font\E="\fontfam/S=5.5:mapping=tex-text" at 11pt
\font\Ew="\fontfam/S=14:color=FFFFFF;mapping=tex-text" at 11pt
\font\EwOrd="\fontfam/S=14:color=FFFFFF;mapping=tex-text;+ordn" at 11pt
\font\it="\fontfam/S=9/I:mapping=tex-text" at 16pt
\font\verbfont="Andale Mono WT J:color=000066" at 13pt

\def\slide#1{\vfill\eject
  \line{\hss \special{x:rulecolor=000066}\vrule height 30pt width \width\special{x:rulecolor}\hss}
  \vskip-26pt\line{\Cw \runheadtext\hfil}
  \vskip16pt\line{\B #1\hfil}}

\def\item{\par\kern\smallskipamount
  \C \noindent \hbox to \parindent{\hfil • }\hangindent\parindent \hangafter1 }
\def\subitem{\par\D \noindent \hbox to 2\parindent{\hfil • }\hangindent2\parindent \hangafter1 }

% adjust the TeX and LaTeX logos to look better in Adobe Garamond Pro / Brioso Pro
\def\TeX{\leavevmode$\smash{\hbox{T\kern-.1667em\lower.5ex\hbox{E}\kern-.125em X}}$}
\def\LaTeX{\leavevmode$\smash{\hbox{L\raise.4ex\hbox{\kern-.25em
   \special{x:gsave}\special{x:scale .72 .72}A\special{x:grestore}\kern-.3333em}\TeX}}$}

% define the XeTeX logo
\def\reflect#1{{\setbox0=\hbox{#1}\rlap{\kern0.5\wd0
  \special{x:gsave}\special{x:scale -1 1}}\box0 \special{x:grestore}}}
\def\XeTeX{\leavevmode$\smash{\hbox{X\lower.5ex
  \hbox{\kern-.125em\reflect{E}}\kern-.1667em \TeX}}$}

\def\TeXgX{\leavevmode$\smash{\hbox{\TeX
  \lower.5ex\hbox{\kern-.15em G}\kern-.1667em X}}$}

% for literal text
\catcode`\|=\active
\def|{\begingroup\def\do##1{\catcode`##1=12 }\dospecials \verbatim}
\def\verbatim#1|{\verbfont #1\endgroup}

\def\red{\aftergroup\resetcolor
  \special{color push rgb 0.4 0.0 0.0}}
\def\green{\aftergroup\resetcolor
  \special{color push rgb 0.0 0.4 0.0}}
\def\resetcolor{\special{color pop}}
\catcode`\•=\active \def•#1•{{\red #1}}
\catcode`\«=\active \def«#1»{{\green #1}}


\footline={%
  \vbox to 0pt{\vss\line{\hss \special{x:rulecolor=000066}\vrule height 15pt width \width\special{x:rulecolor}\hss}
   \vskip-29pt\line{\Ew TUG2005 Conference
   \hfil Wuhan, China, August 2005}\vskip-9pt}}

\vglue1in
\special{x:gsave}
\special{x:colorshadow(2,-2),0.8,00006666}
\centerline{\A \XeTeX: the Multilingual Lion}
\special{x:grestore}
\medskip
\centerline{\B \TeX\ meets Unicode and smart fonts}
\bigskip
\bigskip
\centerline{\D Jonathan Kew}
\smallskip
\centerline{\D SIL International}
\bigskip
\centerline{\E August 23, 2005}

\font\T="TeXLogo:color=000099" at 125pt
\vbox to 0pt{\kern-1.2in
\special{x:gsave}
\special{x:colorshadow(2,-2),0.8,00006666}
\rightline{\T A\kern.5in}
\special{x:grestore}
\vss}
\vbox to 0pt{\kern-1.25in
\leftline{\kern.5in\XeTeXpicfile "unicode-book.jpg" width 1.3in}
\vss}

\def\section#1{\def\runheadtext{#1}}
\section{An introduction to \XeTeX}

\slide{What is \XeTeX?}
\item \TeX\ typesetting engine
\subitem including e-\TeX\ extensions
\=
\item Supporting the Unicode character set
\subitem inherently multilingual/multiscript typesetting system
\subitem greatly simplifies language support at macro level
\=
\item Using modern font technologies
\subitem TrueType, OpenType (all fonts supported by platform)
\=
\item With “smart rendering” support
\subitem Apple Advanced Typography
\subitem OpenType Layout features
\subitem for typographic features and complex scripts

\section{Unicode support}

\slide{Multilingual typesetting with \TeX}
\item Text input
\subitem escape sequences for non-ASCII characters
\subitem multiple 8-bit and double-byte codepages
\subitem use of active characters
\subitem preprocessors for complex scripts 
\=
\item Font support
\subitem fonts limited to 256 glyphs
\subitem custom-encoded fonts with specific glyph sets
\subitem many different font encodings in use
\=
\item All tied together via complex \TeX\ macros
\subitem difficult to understand and extend
\subitem difficult to integrate with other packages

\slide{Traditional \TeX\ input conventions}
\item Input text is ASCII (or 8-bit codepage)
\bigskip
\font\example="Gentium:color=660000;mapping=tex-text" at 20pt
\font\dn="Devanagari MT:color=660000" at 20pt
\halign{\hfil#\hfil\kern20pt&\hfil\example#\hfil\kern20pt&\it#\hfil\cr
\noalign{\hrule\smallskip}
\it Source text&\it Typeset output&\it Notes\cr
\noalign{\smallskip\hrule\smallskip}
|\'{a}|&á&typical accent command\cr
|\c{c}|&ç\cr
|\aa|&å\cr
|---|&---&ligature in typical \TeX\ fonts\cr
|$\alpha$|&α&math mode symbol\cr
|{\dn acchaa}|&\char32$\smash{\hbox{\dn अच्छा}}$\char32&using custom preprocessor\cr
}

\slide{Typesetting Unicode text with \XeTeX}
\item Accented characters
\subitem many more than in any legacy codepage
\bigskip
\line{\hfil
\vtop{\obeylines\parindent0pt\hsize=.6\hsize
|\halign{#\hfil\quad&|
| #\hfil\cr|
|dan&    dan\cr|
|dubok&  dubok\cr|
|džabe&  đak\cr|
|džin&   džabe\cr|
|Džin&   džin\cr|
|đak&    Džin\cr|
|Evropa& Evropa\cr}|
}\hfil
\vtop{\hsize=.4\hsize
\font\txt="Gentium:color=660000" at 20pt \txt
\halign{#\hfil\quad&#\hfil\cr
dan&    dan\cr
dubok&  dubok\cr
džabe&  đak\cr
džin&   džabe\cr
Džin&   džin\cr
đak&    Džin\cr
Evropa& Evropa\cr}
}\hfil}

\slide{Typesetting Unicode text with \XeTeX}
\item CJK ideographs
\subitem they’re just more characters, no special effort required
\font\han="STSong:color=660000" at 24pt
\bigskip
\line{\hfil
\vtop{\obeylines\parindent0pt\hsize=.7\hsize
|\font\han="STSong" at 16pt|
|\font\rom="Gentium" at 8pt|
|\def\hc#1#2{\vtop{\hbox{\han #1}|
| \hbox{\kern10pt\rom #2}}}|
|\vtop{\hc{書く}{ka-ku}|
| \hc{最も}{motto-mo}|
| \hc{最後}{sai-go}|
| \hc{働く}{hatara-ku}|
| \hc{海}{umi}}|}\hfil
\vtop{\hsize=.3\hsize
\font\rom="Gentium:color=006600" at 16pt
\def\hc#1#2{\vtop{\hbox{\han #1}
 \hbox{\kern10pt\rom #2}}}
\vtop{\hc{書く}{ka-ku}
 \hc{最も}{motto-mo}
 \hc{最後}{sai-go}
 \hc{働く}{hatara-ku}
 \hc{海}{umi}}
}\hfil}

\slide{Typesetting Unicode text with \XeTeX}
\item Complex scripts
\subitem just simple character data in the source file

\line{\vtop{\hsize=0.6\hsize \obeylines%
|\c 1|
|\s ‭دنيا ‭جي ‭پيدائش|
|\p|
|\v ‭1 ‭شروعات ‭۾ ‭خدا ‭زمين ‭۽|
|‭آسمان ‭کي ‭پيدا ‭ڪيو. ‭|
|\v ‭2 ‭ان ‭وقت ‭زمين ‭بي​ترتيب|
|‭۽ ‭ويران ‭هئي. ‭اونهي ‭سمنڊ|
|جو ‭مٿاڇرو ‭اوندهہ ‭سان ‭ڍڪيل ‭هو|
|‭۽ ‭پاڻئَ ‭جي ‭مٿان ‭خدا|
|جي ‭روح ‭ڦيرا ‭پئي ‭ڪي|
|\v ‭3 ‭تڏهن ‭خدا ‭حڪم ‭ڏنو ‭تہ ‭”روشني|
|‭ٿئي.“ ‭سو ‭روشني ‭ٿي ‭پيئي. ‭|
}\hfil
\vtop{\hsize=0.35\hsize \rightskip=0pt
\font\sfont="Geeza Pro Bold" at 14pt
\font\txfont="Scheherazade-AAT:Alternate forms=Sindhi meem" at 15pt \txfont
\font\cfont="Geeza Pro Bold" at 32pt
\font\vfont="Scheherazade" at 11pt
\baselineskip=16pt \lineskiplimit=-10pt
\line{\hfil \special{x:gsave}\special{x:scale 1.25 1.25}\special{x:textcolor=660000}}\nointerlineskip
\centerline{\sfont دنيا جي پيدائش}
\smallskip
\def\v #1 {\leavevmode\raise3pt\hbox{\vfont #1}\kern2pt}
\setbox0=\hbox{\cfont ١\kern2pt}
\noindent\beginR\hangindent-\wd0 \hangafter -2
\setbox0=\hbox{\kern-\wd0\lower1.1\baselineskip\box0}\dp0=0pt \box0
\v ١ شروعات ۾ خدا زمين ۽ آسمان کي پيدا ڪيو. 
\v ٢ ان وقت زمين بي​ترتيب ۽ ويران هئي. اونهي سمنڊ
جو مٿاڇرو اوندهہ سان ڍڪيل هو ۽ پاڻئَ جي مٿان خدا
جي روح ڦيرا پئي ڪي
\v ٣ تڏهن خدا حڪم ڏنو تہ ”روشني ٿئي.“ سو روشني ٿي پيئي. 
}\special{x:grestore}\special{x:textcolor}}

\slide{Typesetting Unicode text with \XeTeX}
\item Vertical text \special{x:textcolorpush}\special{x:textcolor=888888}(not fully supported)\special{x:textcolorpop}
\font\mon="Code2000:script=mong;color=660000" at 18pt

\line{\vtop{\hsize=0.8\hsize \obeylines%
|\font\mon="Code2000:script=mong" at 18pt|
|\setbox0=\vbox{|
| \hsize=3.6in \baselineskip=20pt|
| \parindent=-12pt \leftskip=12pt|
|\revpar \mon|
|ᠥᠪᠥᠷ ᠮᠣᠩᠭᠣᠯ ᠤᠨ ᠡᠯ᠎ᠡ ᠴᠢᠭᠤᠯᠭᠠᠨ ᠤ ᠨᠡᠷᠡᠰ ᠦᠨ ᠢᠷᠡᠯᠲᠡ|
|ᠥᠪᠥᠷ ᠮᠣᠩᠭᠣᠯ ᠤᠨ ᠡᠯ᠎ᠡ ᠴᠢᠭᠤᠯᠭᠠᠨ ᠤ ᠨᠡᠷᠡᠰ ᠦᠨ ᠢᠷᠡᠯᠲᠡ|
|ᠥᠪᠥᠷ ᠮᠣᠩᠭᠣᠯ ᠤᠨ ᠡᠯ᠎ᠡ ᠴᠢᠭᠤᠯᠭᠠᠨ ᠤ ᠨᠡᠷᠡᠰ ᠦᠨ ᠢᠷᠡᠯᠲᠡ|
|ᠥᠪᠥᠷ ᠮᠣᠩᠭᠣᠯ ᠤᠨ ᠡᠯ᠎ᠡ ᠴᠢᠭᠤᠯᠭᠠᠨ ᠤ ᠨᠡᠷᠡᠰ ᠦᠨ ᠢᠷᠡᠯᠲᠡ|
|\par}|
|\special{x:gsave}\special{x:rotate -90}|
|\vskip-\ht0 \box0 \special{x:grestore}|
}\kern10pt\vtop{\kern-30pt
% macro to reverse the order of the lines in a paragraph
\def\revpar{\setbox0=\vbox\bgroup
  \let\savepar=\par \let\par=\endrevpar}
\def\endrevpar{\savepar \global\setbox1=\vbox{}
  \loop \skip255=\lastskip \unskip
  \count255=\lastpenalty \unpenalty \setbox0=\lastbox
    \ifhbox0 \global\setbox1=\vbox{\unvbox1
    \penalty\count255 \vskip\skip255 \box0}
      \repeat \egroup \unvbox1 }
\setbox0=\vbox{
 \hsize=3.6in \baselineskip=20pt \lineskiplimit=-1000pt
 \parindent=-12pt \leftskip=12pt \rightskip=0pt
 \tolerance=1000 \hbadness=1000
 \revpar \mon
 ᠥᠪᠥᠷ ᠮᠣᠩᠭᠣᠯ ᠤᠨ ᠡᠯ᠎ᠡ ᠴᠢᠭᠤᠯᠭᠠᠨ ᠤ ᠨᠡᠷᠡᠰ ᠦᠨ ᠢᠷᠡᠯᠲᠡ
 ᠥᠪᠥᠷ ᠮᠣᠩᠭᠣᠯ ᠤᠨ ᠡᠯ᠎ᠡ ᠴᠢᠭᠤᠯᠭᠠᠨ ᠤ ᠨᠡᠷᠡᠰ ᠦᠨ ᠢᠷᠡᠯᠲᠡ
 ᠥᠪᠥᠷ ᠮᠣᠩᠭᠣᠯ ᠤᠨ ᠡᠯ᠎ᠡ ᠴᠢᠭᠤᠯᠭᠠᠨ ᠤ ᠨᠡᠷᠡᠰ ᠦᠨ ᠢᠷᠡᠯᠲᠡ
 ᠥᠪᠥᠷ ᠮᠣᠩᠭᠣᠯ ᠤᠨ ᠡᠯ᠎ᠡ ᠴᠢᠭᠤᠯᠭᠠᠨ ᠤ ᠨᠡᠷᠡᠰ ᠦᠨ ᠢᠷᠡᠯᠲᠡ
 \par}
\special{x:gsave}\special{x:rotate -90}
\vskip-\ht0
\dimen0=\ht0 \ht0=\wd0 \wd0=\dimen0 \box0
\special{x:grestore}
}\hss}

\slide{A cleaner multilingual solution}
\item All required characters directly represented
\subitem no need for “escape sequences” to access characters not included in the current codepage
\subitem no need to switch between codepages according to the language/script being typeset 
\subitem characters rendered via standard access codes
\=
\item Character/glyph model and modern font rendering technologies
\subitem encoded text represents characters, not glyphs
\subitem complex script behavior separated from the encoded text data,
handled through standard “smart font” technologies

\section{Extending \TeX\ to process Unicode}

\slide{Character codes}
\font\simpchin="STHeiti:color=660000;Text Spacing=Monospaced Text" at 13.0pt
\item Basic character codes are 16-bit
\subitem representing Unicode in the UTF-16 encoding form
\subitem (except when using legacy custom-encoded fonts)
\=
\item Extended \TeX\ primitives
\subitem |\char|, |\chardef| accept numbers up to 65,536
\subitem 4-digit hex notation using |^^^^abcd|\hfil\break\indent
 |\char"5609^^^^6167|\ \ =\ \ {\simpchin \char"5609^^^^6167}
\=
\item What about Unicode characters beyond Plane 0?
\subitem handled using surrogates (UTF-16 representation)
\subitem adequate for rendering
\subitem does not allow full per-character programmability

\slide{Extended \TeX\ code tables}
\item Per-character code tables |\catcode|, |\lccode|, |\uccode|, |\sfcode| enlarged
\subitem “plain \XeTeX” format initializes these tables based on Unicode character set
\font\smp="Gentium:color=660000" at 18pt
\subitem |\lowercase{DŽIN}|\hfil\break\indent{\smp\lowercase{DŽIN}}
\subitem |\uppercase{Esi eyama klɔ míaƒe nuvɔwo ɖa vɔ la}|\hfil\break\indent{\smp\uppercase{Esi eyama klɔ míaƒe nuvɔwo ɖa vɔ la}}
\subitem |\catcode`王=\active \def王{...}|

\slide{Input encodings}
\item By default, input read as Unicode (UTF-8 or UTF-16)
\subitem encoding form automatically detected
\=
\item Non-Unicode input text
\subitem legacy codepages supported via ICU converters
\subitem set codepage of current input file:\hfil\break
|\XeTeXinputencoding "|{\it charset-name}|"|
\subitem set initial codepage for newly-opened input files:\hfil\break
|\XeTeXdefaultencoding "|{\it charset-name}|"|

\slide{Hyphenation patterns}
\item Extended for 16-bit characters
\=
\item Standard hyphenation files are encoding-specific
\subitem modified to load correctly under \XeTeX
\=
\item Simple hyphenation for scripts such as Devanagari
\subitem text is simple character data, no macros, active chars, etc.
\medskip
\vbox{\parindent=2\parindent\obeylines%
|% break before or after any independent vowel|
|1अ1|
|1आ1|
|1इ1|
|% break after any dependent vowel, but never before|
|2ा1|
|2ि1|}

\section{Font support in \XeTeX}

\slide{Host platform fonts}
\item Use any font installed on the host computer
\item |\font| command extended to accept “real” font names
\=
\item |\font\rm="Trebuchet MS" at 16pt \rm Hello World!|
\subitem {\font\rm="Trebuchet MS:color=660000" at 16pt \rm Hello World!}
\item |\font\it="Times Italic" at 16pt \it Hello World!|
\subitem {\font\it="Times Italic:color=660000" at 16pt \it Hello World!}
\item |\font\ch="Apple Chancery" at 16pt \ch Hello World!|
\subitem {\font\ch="Apple Chancery:color=660000" at 16pt \ch Hello World!}
\item |\font\heiti="STHeiti" at 16pt \heiti 你好,武汉!|
\subitem {\font\heiti="STHeiti:color=660000" at 16pt \heiti 你好,武汉!}
\=
\item No TFM files, etc., required to use new fonts!

\slide{Output device support}
\item Output driver uses the same fonts as the typesetting engine
\subitem no font name mapping files required
\=
\item Generate PDF as default output
\subitem there is actually an “extended DVI” (|.xdv|) intermediate
\item Fonts automatically embedded and subsetted

\slide{Support for traditional \TeX\ fonts}
\item TFM files still supported
\subitem required for math fonts to provide precise metrics
\subitem implies non-Unicode data, using character codes 0…255 only
\=
\item PDF back-end supports Type 1 fonts
\subitem uses |.pfb| files in the texmf tree, just like dvips
\subitem no support for bitmap fonts
\subitem currently no |.vf| support

\slide{Font mappings}
\item Traditional \TeX\ keyboarding practices
\subitem typical input:\hfil\break
 |``\TeX''---a typesetting system|
\subitem generates: {\red ``\TeX''---a typesetting system}
\=
\item Font mapping for compatibility\hfil\break
{\obeylines%
|; TECkit mapping for TeX input conventions|
|U+002D U+002D        <> U+2013 ; -- -> en dash|
|U+002D U+002D U+002D <> U+2014 ; --- -> em dash|
|U+0027         <>  U+2019 ; ' -> right single quote|
|U+0027 U+0027  <>  U+201D ; '' -> right double quote|
|U+0022          >  U+201D ; " -> right double quote|
}
\medskip
\subitem generates: {\red “\TeX”—a typesetting system}
\subitem the “font mapping” is associated with a specific \TeX\ font identifier

\slide{More fun with font mappings}
\bigskip
\line{\vbox{\hsize=.6\hsize
{\obeylines%
|\def\SampleText{Unicode -|
|    это уникальный|
| код для любого символа,\\|
| независимо от платформы,\\|
| независимо от программы,\\|
| независимо от языка.}|
|«\font\gen="Gentium"»|
|\gen\SampleText|
|\bigskip|
|•\font\gentrans="Gentium:•|
|  •mapping=cyr-lat-iso9"•|
|\gentrans\SampleText|
}}\hss
\vbox{\hsize=.4\hsize
\leftskip=\rightskip \parindent=0pt \parfillskip=0pt \def\\{\break}
\def\SampleText{Unicode -
    это уникальный
 код для любого символа,\\
 независимо от платформы,\\
 независимо от программы,\\
 независимо от языка.\par}
\font\gen="Gentium:color=006600" at 13pt
\gen\SampleText
\bigskip
\font\gentrans="Gentium:color=660000;
  mapping=cyr-lat-iso9" at 13pt
\gentrans\SampleText
}}

\section{Typographic features}

\slide{AAT font features}
{
\let\DoSpecials=\dospecials \def\dospecials{\DoSpecials \relax \catcode32=10 }
\item Custom AAT features accessed via |\font| command
\item |\font\x="Apple Chancery" at 16pt \x The quick brown fox jumps over the lazy dog.|
\subitem {\font\x="Apple Chancery:color=660000;Smart Swashes=!Line Initial Swashes,!Line Final Swashes" at 16pt \x The quick brown fox jumps over the lazy dog.}
\=
\item |\font\x="Apple Chancery:Letter Case=Small Caps;Design Complexity=Simple Design Level" at 16pt \x The quick…|
\subitem {\font\x="Apple Chancery:Letter Case=Small Caps;Design Complexity=Simple Design Level;color=660000;Smart Swashes=!Line Initial Swashes,!Line Final Swashes" at 16pt \x The quick brown fox jumps over the lazy dog.}
\=
\item |\font\x="Apple Chancery:Design Complexity=Flourishes Set A" at 16pt \x The quick brown fox jumps over…|
\subitem {\font\x="Apple Chancery:Design Complexity=Flourishes Set A;color=660000;Smart Swashes=!Line Initial Swashes,!Line Final Swashes" at 16pt \x The quick brown fox jumps over the lazy dog.}
}

\slide{OpenType: language and script}
\item Fonts may support multiple languages with differing behavior
{\obeylines
|\font\Doulos="Doulos SIL/ICU"|
|\font\DoulosViet="Doulos SIL/ICU•:language=VIT•"|}

\font\Doulos="Doulos SIL/ICU" at 22pt
\font\DoulosViet="Doulos SIL/ICU:language=VIT" at 22pt
\line{\let\p=\par\hfil\vtop{\hsize=.45\hsize\lineskip=-7pt\Doulos
\obeylines \parindent=0pt \leftskip=0pt plus 1in \rightskip=\leftskip \parfillskip=0pt
Unicode cung c•ấ•p một con s•ố• duy nh•ấ•t cho m•ỗ•i ký tự\p}\hfil
\vtop{\hsize=.45\hsize\lineskip=-7pt\DoulosViet
\obeylines \parindent=0pt \leftskip=0pt plus 1in \rightskip=\leftskip \parfillskip=0pt
Unicode cung c•ấ•p một con s•ố• duy nh•ấ•t cho m•ỗ•i ký tự\p}\hfil}

\=
\bigskip

{\obeylines
|\font\Brioso="Brioso Pro"|
|\font\BriosoTrk="Brioso Pro•:language=TRK•"|}
\smallskip
\font\Brioso="Brioso Pro/S=12" at 24pt
\font\BriosoTrk="Brioso Pro/S=12:language=TRK" at 24pt
\line{\let\p=\par\hfil\vtop{\hsize=.45\hsize\Brioso
 \parindent=0pt \leftskip=0pt plus 1in \rightskip=\leftskip \parfillskip=0pt
… gelen •fi•rmaları … tara•fı•ndan …\p}\hfil
\vtop{\hsize=.45\hsize\BriosoTrk
 \parindent=0pt \leftskip=0pt plus 1in \rightskip=\leftskip \parfillskip=0pt
… gelen •fi•rmaları … tara•fı•ndan …\p}\hfil}

\slide{OpenType: language and script}
\item Complex Asian scripts require specific “shaping engines”
\item With no “script tag”, only default Latin features applied
\vadjust{\medskip}\hfil\break |\font\x="Code2000" \x العربي हिन्दी |\hfil\break\indent
{\font\x="Code2000:color=660000" at 20pt \x العربي हिन्दी}
\=
\medskip
\item Must load the font with the appropriate shaping engine
\vadjust{\medskip}\hfil\break |\font\x="Code2000:script=arab" \x العربي|\hfil\break\indent
{\font\x="Code2000:script=arab;color=660000" at 20pt \x العربي}
\vadjust{\bigskip}
\hfil\break |\font\x="Code2000:script=deva" \x हिन्दी|\hfil\break\indent
{\font\x="Code2000:script=deva;color=660000" at 20pt \x हिन्दी}

\slide{OpenType: optional features}
\item Font specification may include feature tags
\subitem |\font\x="Brioso Pro" \x Hello World! 0123456789|\hfil\break\indent
{\font\x="Brioso Pro:color=660000" at 24pt \x Hello World! 0123456789}
\subitem |\font\x="Brioso Pro:+smcp"|\hfil\break\indent
{\font\x="Brioso Pro:+smcp;color=660000" at 24pt \x Hello World! 0123456789}
\subitem |\font\x="Brioso Pro:+sups"|\hfil\break\indent
{\font\x="Brioso Pro:+sups;color=660000" at 24pt \x Hello World! 0123456789}
\subitem |\font\x="Brioso Pro Italic:+onum"|\hfil\break\indent
{\font\x="Brioso Pro Italic:+onum;color=660000" at 24pt \x Hello World! 0123456789}
\subitem |\font\x="Brioso Pro Italic:+swsh,+zero"|\hfil\break\indent
{\font\x="Brioso Pro Italic:+swsh,+zero;color=660000" at 24pt \x Hello World! 0123456789}

\slide{OpenType: optical sizing}

\item OpenType optical families automatically choose correct face for the size used
\subitem Brioso Pro at 7, 10, 18, 24pt sizes:\hfil\break
{\font\x="Brioso Pro:color=660000" at 7pt \x seven}
{\font\x="Brioso Pro:color=660000" at 10pt \x ten}
{\font\x="Brioso Pro:color=660000" at 18pt \x eighteen}
{\font\x="Brioso Pro:color=660000" at 24pt \x twenty-four}
\=
\item Can override with |/S=| modifier on font name
\subitem showing different optical sizes using the same “at size”
\subitem{\font\x="Brioso Pro/S=7:color=660000" at 24pt \x \hbox to 2in{|Brioso Pro/S=7|\hfil}Brioso Pro Caption}
\subitem{\font\x="Brioso Pro/S=10:color=660000" at 24pt \x \hbox to 2in{|Brioso Pro/S=10|\hfil}Brioso Pro Text}
\subitem{\font\x="Brioso Pro/S=18:color=660000" at 24pt \x \hbox to 2in{|Brioso Pro/S=18|\hfil}Brioso Pro Subhead}
\subitem{\font\x="Brioso Pro/S=24:color=660000" at 24pt \x \hbox to 2in{|Brioso Pro/S=24|\hfil}Brioso Pro Display}

\section{Asian-language linebreaking}

\slide{Line-break positions}

\font\simpchin="STHeiti:color=660000;Text Spacing=Monospaced Text" at 13.0pt
%\font\thai="Thonburi:color=660000" at 16pt
\item Line breaking without word spaces
\subitem \TeX\ normally breaks lines at “glue” arising from spaces
\subitem Chinese, Japanese, Thai, etc. do not use word spaces
\iffalse
\subitem {\rightskip0pt \hfuzz\maxdimen\thai โดยพื้นฐานแล้ว, คอมพิวเตอร์จะเกี่ยวข้องกับเรื่องของตัวเลข. คอมพิวเตอร์จัดเก็บตัวอักษรและอักขระอื่นๆ โดยการกำหนดหมายเลขให้สำหรับแต่ละตัว. ก่อนหน้าที่๊ Unicode จะถูกสร้างขึ้น, ได้มีระบบ encoding อยู่หลายร้อยระบบสำหรับการกำหนดหมายเลขเหล่านี้.\par}
\=
\item Use ICU line-break: |\XeTeXlinebreaklocale "th"|
\XeTeXlinebreaklocale "th"
\XeTeXlinebreakskip 0pt plus 2pt
\subitem {\rightskip0pt \thai โดยพื้นฐานแล้ว, คอมพิวเตอร์จะเกี่ยวข้องกับเรื่องของตัวเลข. คอมพิวเตอร์จัดเก็บตัวอักษรและอักขระอื่นๆ โดยการกำหนดหมายเลขให้สำหรับแต่ละตัว. ก่อนหน้าที่๊ Unicode จะถูกสร้างขึ้น, ได้มีระบบ encoding อยู่หลายร้อยระบบสำหรับการกำหนดหมายเลขเหล่านี้.\par}
\else
\subitem {\baselineskip=16pt
\simpchin\rightskip=0pt\hfuzz\maxdimen 基本上,计算机只是处理数字。它们指定一个数字,来储存字母或其他字符。在创造Unicode之前,有数百种指定这些数字的编码系统。没有一个编码可以包含足够的字符:\par}
\=
\item Use ICU line-break algorithm
\subitem find permitted line-break locations according to a specific locale
\subitem |\XeTeXlinebreaklocale "zh"|\vadjust{\vskip2pt}
\XeTeXlinebreaklocale "zh"
\hfil\break {\baselineskip=16pt
\simpchin\rightskip=0pt plus 1fil 基本上,计算机只是处理数字。它们指定一个数字,来储存字母或其他字符。在创造Unicode之前,有数百种指定这些数字的编码系统。没有一个编码可以包含足够的字符:\par}
\fi
\XeTeXlinebreaklocale ""

\slide{Justification}

\item Text without spaces is difficult for \TeX\ to justify
\item Ragged-right setting is one solution
\XeTeXlinebreaklocale "zh"
\subitem {\baselineskip=16pt
\simpchin\rightskip=20pt plus 1fil 基本上,计算机只是处理数字。它们指定一个数字,来储存字母或其他字符。在创造Unicode之前,有数百种指定这些数字的编码系统。没有一个编码可以包含足够的字符:\par}
\=
\item Alternatively, use |\XeTeXlinebreakskip| to introduce glue at each potential break
\XeTeXlinebreakskip 0pt plus 2pt
\subitem {\baselineskip=16pt
\simpchin\rightskip=20pt 基本上,计算机只是处理数字。它们指定一个数字,来储存字母或其他字符。在创造Unicode之前,有数百种指定这些数字的编码系统。没有一个编码可以包含足够的字符:\par}
\=
\item Could also use non-monospaced Latin characters
\font\simpchin="STHeiti:color=660000" at 13.0pt
\XeTeXlinebreakskip 0pt plus 2pt
\subitem {\baselineskip=16pt
\simpchin\rightskip=20pt 基本上,计算机只是处理数字。它们指定一个数字,来储存字母或其他字符。在创造Unicode之前,有数百种指定这些数字的编码系统。没有一个编码可以包含足够的字符:\par}
\XeTeXlinebreaklocale ""

\section{Built-in graphics support}

\slide{QuickTime image support}
\item Many graphic file formats directly supported
\subitem TIFF, JPEG, PNG, BMP, PICT, GIF, TGA, Photoshop, …
\subitem |\setbox0=\hbox{\XeTeXpicfile "mypic.jpg"}|
\=
\item Optional keywords to modify image
\subitem scaled, xscaled, yscaled, width, height, rotated
\smallskip
\centerline{\hbox{\XeTeXpicfile "unicode-book.jpg" scaled 100}\qquad
 \hbox{\XeTeXpicfile "unicode-book.jpg" scaled 100 xscaled 2000}\qquad
 \hbox{\XeTeXpicfile "unicode-book.jpg" scaled 100 rotated 90}}
\=
\item Image width and height available to \TeX\ engine
\item Can use via \LaTeX\ and Con\TeX t commands

\slide{PDF documents}
\item Beware: QuickTime graphic importer accepts PDF
\subitem but renders as raster image at screen resolution!
\item Use alternative command for true PDF inclusion
\subitem |\XeTeXpdffile "xetex-intro-slides.pdf" page 1|\hfil\break| scaled 400|
\smallskip
\centerline{\setbox0=\vbox{\hbox{\XeTeXpdffile "xetex-wuhan.pdf" page 1 scaled 400}}%
  \wd0=205bp \ht0=154bp \box0}

\section{\LaTeX\ and Con\TeX t support}

\slide{fontspec.sty by Will Robertson}
\item Simple specification of native OS X fonts in \LaTeX
\item Integrates \XeTeX\ font access with \LaTeX\ commands
\subitem setting the default document fonts\hfil\break
\vbox{\parindent=0pt \hsize=.9\hsize \obeylines%
|\usepackage{fontspec}|
|\setromanfont{Adobe Garamond Pro}|
|\setmonofont[Scale=0.8]{Andale Mono}|}
\medskip
\=
\subitem on-the-fly font and feature changes\hfil\break
\vbox{\parindent=0pt \hsize=.9\hsize \obeylines%
|Welcome to Wuhan,|
|{\addfontfeature{LetterCase=SmallCaps}China}|}\hfil\vadjust{\medskip}\break
\indent {\red Welcome to Wuhan,
\font\x="Brioso Pro/S=9:+smcp" at 18pt \x China}\hfil\vadjust{\smallskip}\break
\vbox{\parindent=0pt \hsize=.9\hsize \obeylines%
|August 25{\addfontfeature{VerticalPosition=Superior}th}|\kern-20pt}\hfil\vadjust{\medskip}\break
\indent {\red August 25\font\x="Brioso Pro/S=9:+sups" at 18pt \x th}


\slide{xunicode.sty by Ross Moore}
\item Support for standard \LaTeX\ input of many special characters when using Unicode fonts
\subitem accent commands, named characters, etc., mapped to Unicode values for font access
\subitem does not handle dashes, quotes (use |tex-text| font mapping)
\item Allows many non-Unicode \LaTeX\ documents to be processed using Unicode fonts

\slide{Using Con\TeX t with \XeTeX}

\item Reportedly works fairly readily, but not pre-configured “out of the box”
\subitem see |http://www.contextgarden.net/XeTeX|
\=
\item Use \XeTeX\ font names and features in Con\TeX t typescripts and other font definitions
\subitem see |http://www.contextgarden.net/Fonts_in_XeTeX|
\bigskip
\line{\hfil
\vtop{\obeylines\parindent20pt\hsize=.6\hsize
|\definedfont["Hoefler Text:|
\quad|mapping=tex-text;|
\quad|Style Options=Engraved Text;|
\quad|Letter Case=All Capitals;|
\quad|color=229966" at 32pt]|
|Big Title|
}\hfil
\vtop{\hsize=.4\hsize
\font\bigtitle="Hoefler Text:Letter Case=All Capitals;
  Style Options=Engraved Text;color=229966" at 32pt
\vskip24pt
\bigtitle Big Title
}\hfil}

\section{Status and future directions}

\slide{What might be next for \XeTeX?}
\item Ongoing bug-fixes and minor features
\=
\item Enhanced PDF back-end
\subitem leverage improved PDF support in Mac OS X 10.4
\subitem new |xdv2pdf| driver based on |dvipdfmx|
\subitem integration with pdf\TeX\ output routine
\=
\item True Unicode math support
\subitem requires extensions to |\mathchar| etc., and underlying structures
\subitem also requires extended (at least 16-bit) font metric format
\font\gen="Gentium" at 17pt
\subitem may be possible to make use of code from $\smash{\hbox{\gen Ω}}$
\=
\item \XeTeX\ for non-Mac OS platforms
\subitem working towards integration with \TeX\ Live sources

\slide{Questions… and answers?}
\item Contact information
\subitem |mailto:jonathan_kew@sil.org|
\item \XeTeX\ web site and mailing list
\subitem |http://scripts.sil.org/xetex|
\subitem |http://tug.org/mailman/listinfo/xetex|
\subitem |svn://scripts.sil.org/xetex/TRUNK|

\catcode"200C=\active \def^^^^200c{\kern1sp}
\vbox to 0pt{\kern30pt
\special{x:gsave}
\special{x:colorshadow(2,-2),0.8,00006666}
\rightline{\T A}
\special{x:grestore}
\vss}
\vbox to 0pt{\kern24pt
\baselineskip=17pt \lineskiplimit=-1000pt \hsize=.8\hsize \rightskip=0pt
\XeTeXlinebreaklocale "th_TH"
\font\ethiopic="Code2000:color=660000" at 14pt
\font\arabic="Geeza Pro:color=006600" at 14pt
\font\chinese="STKaiti:color=000066" at 14pt
\font\latin="Gentium:color=333300" at 14pt
\font\georgian="Code2000:color=003333" at 14pt
\font\greek="Gentium:color=330033" at 14pt
\font\hebrew="Lucida Grande:color=660000" at 14pt
\font\hindi="Sanskrit 2003:script=deva;color=006600" at 14pt
\font\japanese="Hiragino Mincho Pro W3:color=000066" at 14pt
\font\korean="AppleMyungjo Regular:color=333300" at 14pt
\font\farsi="Geeza Pro:color=003333" at 14pt
\font\cyrillic="Gentium:color=330033" at 14pt
\font\thai="Thonburi:color=006600" at 14pt
\ethiopic ዩኒኮድ ምንድን ነው?
\beginR \arabic ما هي الشفرة الموحدة "يونِكود"؟\endR\ 
\chinese 什麽是Unicode(統一碼/標準萬國碼)?
\latin Što je Unicode?
\georgian რა არის უნიკოდი?
\greek Τί εἶναι τὸ Unicode;
\beginR\hebrew ‏מה זה יוניקוד?\endR\ 
\hindi यूनिकोड क्या है?
\latin Hvað er Unicode?
\japanese ユニコードとは何か?
\korean 유니코드에 대해?
\beginR \farsi يونی‌کُد چيست؟\endR\ 
\cyrillic Что такое Unicode?
\thai Unicode คืออะไร?
\ethiopic ዩኒኮድ እንታይ ኢዩ?
\par\vss}

\bye
